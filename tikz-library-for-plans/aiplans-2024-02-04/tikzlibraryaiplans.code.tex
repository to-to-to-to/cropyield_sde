% author: Mario Schmautz <mario.schmautz@uni-ulm.de>
% modifications after 1.0 by: Pascal Bercher <pascal.bercher@anu.edu.au>

% 
% Please read the "IMPORTANT NOTES" after the change log that
% warns you of some pitfalls in using the library. This is likely
% to save you hours of debugging. :)
% 

% version: 1.07 (2024-02-02)
% 
% change log:
% 
% 1.06 -> 1.07 // all thanks to chatGPT 4.0!
% - Previously, the very last precondition text was always a bit more to the left than
%   the others. This is now resolved (though the code for that changed substantially)
% - Now, schema names recognize line breaks. This is important for long action names. 
%   A canonical way to exploit this is by putting the parameter list into a second line.
% - removed the 'stage' commands; I found them pointless and confusing.^^
% 
% 1.05 -> 1.06
% - changed the color of XXX from "cyan" to "main_color", which allows to easily change it
%   without having to manipulate this file.
% 
% 1.04 -> 1.05
% - made the variable names of the stage command consistent: now they are "init" and "goal"
%   (before they were "start" and "goal"; also okay would have been "start" and "end", but
%    this synonym has not been added)
% - added a note explaining how to use the library accordingly
% 
% 1.03 -> 1.04
% - added commands for adding ordering constraints
% 
% 1.02 -> 1.03
% - added parameters to stage command: instead of displaying 'start' and 'goal', those
%   strings can now be specified
% - added feature request list
% 
% 1.01 -> 1.02
% - added "raise = .25 ex," to all text decorations (in actions), such that the
%   precondition and effect texts are not directly on the lines, but a bit above them
% 
% 1.00 --> 1.01
% - added version number and history of changes
% - moved required libraries from examples in here
% - added library fit, which enables fitting nodes together (useful for methods)
% 


\usetikzlibrary{shapes.geometric, calc, math, decorations.text, positioning, arrows.meta, fit}

\newcounter{aiplans@numeffs} % number of effects in the current action
\newcounter{aiplans@numpres} % number of preconditions in the current action

\tikzset{aiplans action body/.style = {draw}}
\tikzset{aiplans action limb/.style = {draw}}
\tikzset{aiplans emph/.style = {main_color, text=white}}

\newcommand{\aiplans@body}[1]{
  \tikzset{aiplans action body/.style = {#1}}
}

\newcommand{\aiplans@limb}[1]{
  \tikzset{aiplans action limb/.style = {#1}}
}

% Definition of the key-value pairs that can be used with \scheme and \action.
\pgfkeys{
  /aiplans/.is family, /aiplans,
  default/.style = {
    text = {},
    effs = {},
    pres = {},
    height = 10ex,
    width = 12em,
    pre length  = 12em,
    pre lengths = {},
    eff length  = 12em,
    eff lengths = {}},
  pre/.style = {pres = { {#1}}}, % the space is important for \foreach
  eff/.style = {effs = { {#1}}},
  limb length/.style = {pre length = #1, eff length = #1},
  % text/.estore in = \aiplans@text, % probelematic for \textbf \emph etc.
  text/.store in = \aiplans@text,
  pres/.store in = \aiplans@pres,
  effs/.store in = \aiplans@effs,
  height/.estore in = \aiplans@height,
  width/.estore in = \aiplans@width,
  pre length/.estore in = \aiplans@prelength,
  pre lengths/.estore in = \aiplans@prelengths,
  eff length/.estore in = \aiplans@efflength,
  eff lengths/.estore in = \aiplans@efflengths,
}

\pgfkeys{
  /aiplans,
  body/.code = {\aiplans@body{#1}},
  limb/.code = {\aiplans@limb{#1}}
}


% \scheme defines a parameterized scheme that can be instantiated
% with \action.
% #1 name
% #2 number of arguments
% #3 key-value pairs
\newcommand{\scheme}[3]{
  \pgfkeys{/aiplans, #1/.style n args =#2{#3}}
}



% for causal links
% #1 from (coordinate)
% #2 to (coordinate)
% #3 tikz edge/arrow
\newcommand{\link}[3]{
  \node (aiplans circle from) at (#1) [circle,fill,inner sep=0.1em] {};
  \node (aiplans circle to)   at (#2) [circle,fill,inner sep=0.1em] {};
  \draw (aiplans circle from) #3 (aiplans circle to);
}

% for causal links
% #1 from (node name)
% #2 to (node name)
\newcommand{\ordering}[2]{
  \draw[bend left,->] (#1) to node[midway,above] {$<$} (#2);
}

% for causal links
% #1 from (node name)
% #2 to (node name)
\newcommand{\orderingLeft}[2]{
  \ordering{#1}{#2}
}

% for causal links
% #1 from (node name)
% #2 to (node name)
\newcommand{\orderingRight}[2]{
  \draw[bend right,->] (#1) to node[midway,above] {$<$} (#2);
}


% #1 name
% #2 key-value pairs
\def\action#1#2{
  \def\aiplans@name{#1}

  \pgfkeys{/aiplans, default, #2}

  \setcounter{aiplans@numpres}{0}
  \foreach \x in \aiplans@pres {
    \stepcounter{aiplans@numpres}
  }

  \setcounter{aiplans@numeffs}{0}
  \foreach \x in \aiplans@effs {
    \stepcounter{aiplans@numeffs}
  }

  \node [
    align=center,
    minimum height = \aiplans@height,
    minimum width  = \aiplans@width,
    aiplans action body, draw, rectangle
  ] (\aiplans@name) {\aiplans@text};
  
  
  
  % PRECONDITIONS
  
  \foreach \pre [count=\k] in \aiplans@pres{
    \tikzmath{
      \s = 1/\value{aiplans@numpres};
      \r = \s * (\k-1) + \s/2;
    }

    \coordinate (\aiplans@name/pre/start/\k) at ($(\aiplans@name.north west)!\r!(\aiplans@name.south west)$);

    \def\aiplans@length{\aiplans@prelength}
    \foreach \x / \y in \aiplans@prelengths {
      \if\x\k
        \xdef\aiplans@length{\y}
      \fi
    }
    
    % place a coordinate we can use as a name!
    \coordinate (\aiplans@name/pre/\k) at ($(\aiplans@name/pre/start/\k)+(-\aiplans@length,0)$);

    % Draw the precondition line 
    \draw[aiplans action limb] (\aiplans@name/pre/\k) -- (\aiplans@name/pre/start/\k);
    
    % Draw the precondition text
    \node at (\aiplans@name/pre/start/\k) [anchor=east,xshift=.5mm,yshift=1.5mm] {\pre};
  }
  

  
  % EFFECTS
  
  \foreach \eff [count=\k] in \aiplans@effs {
  
    \tikzmath{
      \s = 1/\value{aiplans@numeffs};
      \r = \s * (\k-1) + \s/2;
    }
  
    \coordinate (\aiplans@name/eff/start/\k) at
      ($(\aiplans@name.north east)!\r!(\aiplans@name.south east)$);
  
    \def\aiplans@length{\aiplans@efflength}
    \foreach \x / \y in \aiplans@efflengths {
      \if\x\k
        \xdef\aiplans@length{\y}
      \fi
    }
  
    % place a coordinate we can use as a name!
    \coordinate (\aiplans@name/eff/\k) at ($(\aiplans@name/eff/start/\k)+(\aiplans@length,0)$);
  
    % Draw the line for effect
    \draw[aiplans action limb] (\aiplans@name/eff/start/\k) -- (\aiplans@name/eff/\k);
  
    % Place the text using a node, correctly positioned
    \node at (\aiplans@name/eff/start/\k) [anchor=west,xshift=-.5mm,yshift=1.5mm] {\eff};
  }
}
\endinput
