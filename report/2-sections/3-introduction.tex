\section*{Introduction}
Crop yield prediction is a highly complex and intricate problem due to its nature as a multivariate random dynamic system. This complexity stems from the multitude of factors influencing crop yields, which encompass weather conditions, soil types, and variations in both genotype and phenotype. These variations alone introduce nearly thousands of individual factors that need to be considered.

The factors influencing crop yield are numerous and varied. Weather-related factors include temperature, precipitation, humidity, and sunlight, all of which can fluctuate widely and have significant impacts on crop growth. Soil type adds another layer of complexity, as different soils have varying capacities for water retention, nutrient availability, and root support. Genotype and phenotype variations introduce additional complexity, as different crop varieties and their specific traits interact differently with environmental conditions.

When scaled to large datasets, the computational demands of analyzing and predicting crop yields become immense. Traditional and deep machine learning models face significant challenges in handling the high dimensionality of data, which includes thousands of genotype, phenotype, and environmental variable features. Processing such large volumes of data is computationally expensive, requiring substantial resources and time.

Moreover, these factors are not only numerous but also interdependent and often hidden, adding to the difficulty of making accurate predictions. This complexity is akin to predicting stock exchange rates, where numerous invariant and hidden factors, such as market trends, economic indicators, and investor behaviors, must be accounted for to make reliable predictions. In both cases, the challenge lies in accurately modeling a system influenced by a wide array of dynamic and interrelated variables.

In summary, crop yield prediction requires advanced computational approaches to manage and analyze the vast and complex datasets involved. Innovative methods that can efficiently handle the high dimensionality and dynamic nature of the data are essential to improving prediction accuracy and making effective use of the available information.
