\chapter{Reasonable Title for Main Content}\label{chap:content}

This chapter holds your contributions. Depending on your exact topic, you might use only a single chapter for your actual contributions, or several. Even having several is not unusual. Discuss your proposal with your supervisor(s) and propose descriptive titles. 

The following sections give additional advice that is specifically tailored to students who are new to either \LaTeX{} or scientific writing. I \emph{strongly} suggest that you read the entire document carefully! Also use it as checklist after you started writing and after finishing.


\section{Abstract Advice}

\begin{itemize}
 \item \textbf{Start early.} Writing a report is hard and takes time. More than you think. \emph{Hofstadter's Law:} ``It always takes longer than you expect, even when you take into account Hofstadter's Law.'' -- \emph{So start early!}
 \item \textbf{Read and check your work!} Each \LaTeX{} editor has a spellchecker. Use it!! Read your work \emph{carefully} and \emph{multiple times} before showing it to your supervisor. And \emph{please} read the next sections before doing so! I \emph{guarantee} that it explains errors that you do! Prevent them! Use the next section as checklist!
 \item \textbf{Involve your supervisor!} Don't be afraid to reach out to your supervisor(s)! It's quite literally his/her/their job to supervise you and help you succeed! :) So make sure you get what you need to be successful, don't hold back. But make it easy for them. Most academics are overworked...
 \item \textbf{Choose your title page.} This template has two distinct title pages. Choose the one you like more by setting up the configuration file accordingly. You could even change the template if you like. It's not to restrict you but to save you time.

\end{itemize}


\pagebreak %
\section{Building the PDF}

The file mainfile.tex needs to be compiled to obtain the desired output. 

The easiest way is however so simply type ``make'', which will compile everything for you: The makefile is set up to use \emph{latexmk} by default. This very useful commandline tool works on all standard operating systems: Linux, macOS, and Windows. The installation overhead is minimal, and on Linux and macOS it should even already be installed. Check it out here: \url{https://mg.readthedocs.io/latexmk.html} -- this is the preferred option since it's the most convenient and takes the least time to compile a document. This is because \LaTeX{} documents might have to be compiled up to \emph{five} times! But latexmk knows the exact amount of compilations required based on compiled files available.

Instructions on how to use the makefile:
\begin{itemize}
  \item \textbf{Type ``make''}. Just calling ``make'' without arguments will invoke the build tool \emph{latexmk} in its \emph{online mode}, which automatically updates your PDF every single time you save an updated tex file. (So the terminal you used to invoke make will remain forever idle.) Note that when it encounteres a compilation error it often requires a terminal input from you. In this case, fix the \LaTeX{} error(s) and then type X to continue. \textbf{This is the preferred mode! So just type ``make''.}
  \item \textbf{Type ``make mk''}. Calling make with the argument \emph{mk} will also invoke \emph{latexmk}, but without its online mode. So you will have to call it each time you want to have an updated PDF.
  \item \textbf{Type ``make all''}. Calling make with the argument ``all'' will invoke the \LaTeX{} and bibtex compiler manually the maximal amount of times to compile the document. This is more time-intense than simply letting latexmk make its job. Not recommended.
  \item  \textbf{Type ``make quick''}. Calling make with the argument ``quick'' will simply compile the document a single time with the \LaTeX{} compiler. This is quick, but is mostly not enough to show the updated PDF. Not recommended.
  \item \textbf{Type ``make clear''} (or ``make clean''). Calling make with the argument ``clear'' or ``clean'' will delete all temporary files, such as .aux, .log and so on. This is very rarely required. However, sometimes when you produce very wrong code, \LaTeX{} compilation simply fails until you delete all these files from previous compilations. In such a rare scenario, ``make clear'' will help.
\end{itemize}

Alternatively you should also be able to copy the template onto Overleaf, which will then deal with the compilation for you. Once you copied it over, make sure everything works (i.e., that the PDF is created successfully) before you make any changes. The template as provided should already work (=compile) without the need to make \emph{any changes}.

\textbf{Advice:} Please do not ignore any warnings! They usually should be fixed...